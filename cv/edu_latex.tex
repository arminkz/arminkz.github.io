\documentclass{resume} % Use the custom resume.cls style
\usepackage{multicol}
\usepackage{scrextend}

\usepackage[left=0.4 in,top=0.4in,right=0.4 in,bottom=0.4in]{geometry} % Document margins
\newcommand{\tab}[1]{\hspace{.2667\textwidth}\rlap{#1}} 
\newcommand{\itab}[1]{\hspace{0em}\rlap{#1}}
\name{Armin Kazemi} % Your name
% if you do not have a website, sub your github instead.
\address{+98 9025500510 \\ Tehran, Iran} 
\address{\href{mailto:arminkz3@gmail.com}{arminkz3@gmail.com} \\ \href{arminkz.github.io}{arminkz.github.io}}  %

\begin{document}


%----------------------------------------------------------------------------------------
\begin{rSection}{Education}


{\bf BSc. Computer Engineering}, \href{aut.ac.ir}{Amirkabir University of Technology} \hfill {2015 - 2020}


\begin{itemize}
    \itemsep -3pt {} 
     \item Cumulative GPA: \textbf{17.73/20  (3.65/4)} 
     \item Selected Courses:
	\small{Computer Engineering Project  (20 / 20) | Artificial Intelligence \& Expert Systems (20 / 20)  \\ Data Mining (20 / 20) | Algorithm Design (20 / 20) | Advanced Computer Programming (20 / 20)}
 \end{itemize}

\end{rSection}


\begin{rSection}{RESEARCH INTERESTS}

\begin{multicols}{3}
\begin{itemize}
    \itemsep -3pt {} 
     \item Computer Graphics
     \item Computer Vision
     \item GANs
     \item Visualization
     \item Machine Learning
     \item Deep Learning
 \end{itemize}
 \end{multicols}

\end{rSection}


\begin{rSection}{PUBLICATIONS}

\begin{itemize}
    \itemsep -3pt {}
    \item A. Kazemi, N. Gholipour, H. Faragardi, A. Abderezaei and H. Fotouhi, \textbf{“Optimizing Sink Node Placement in Wireless Sensor Networks,”} Sensors 2021. (Under prep.)
 \end{itemize}

\end{rSection}


\begin{rSection}{NOTABLE PROJECTS}
\vspace{-1.25em}


\item \textbf{Detecting COVID Hotspots and Crowdedness in Public Places} {IoT, Machine Learning, MQTT}
\begin{addmargin}[2em]{2em}
As my BSc. project, Implemented a Internet of Things solution in order to detect crowded areas by analyzing wireless (Wi-Fi and Bluetooth) footprint from smartphones. Moreover, utilizing Machine Learning to predict crowdedness in future days. \href{https://github.com/arminkz/crowdedness-detection}{(view on Github)}
\end{addmargin}


\item \textbf{Chess Bot} {Python, Tensorflow, Convolutional Neural Networks, Computer Vision}
\begin{addmargin}[2em]{2em}
Created a Computer Vision Algorithm to detect chessboard on the screen, Implemented a CNN to extract current position on the Chessboard then feeding the FEN (Forsyth–Edwards Notation) to a chess engine and automatically play the game. \href{https://github.com/arminkz/ChessBot}{(view on Github)}
\end{addmargin}


\item \textbf{Reversi AI}  {Java, Classic Artificial Intelligence, Minimax with A/B Pruning}
\begin{addmargin}[2em]{2em}
Created an Artificial Intelligence for the Reversi (also known as Othello) boardgame, which uses Minimax with A/B pruning, also adapted some Machine Learning techniques for better evaluation of game positions. \href{https://github.com/arminkz/Reversi}{(view on Github)}
\end{addmargin}

\item \textbf{Eye Tracking in VR headsets} {Python, OpenCV}
\begin{addmargin}[2em]{2em}
Implemented a Computer Vision algorithm for detecting user's gaze point in VR headsets using a embeded camera behind the VR lens. \href{https://github.com/arminkz/EyeTrackingVR}{(view on Github)}
\end{addmargin}


\item \textbf{Persian News Search Engine} {Python, Angular, TF-IDF, Inverted Index, KMeans, Crawler}
\begin{addmargin}[2em]{2em}
Implemented a Persian language news search engine. Including a front-end UI, a Crawler and the engine itself. The engine utilizes Mini-Batch-KMeans for large scale clustering and TF-IDF algorithm for intra-cluster searching. Moreover, some stemming techniques has been put to use.
\href{https://github.com/kianelbo/newsKAP}{(view on Github)}
\end{addmargin}

\item \textbf{PoorCraft} {Java, Isometric Game Engine, Strategic Game, Network Game}
\begin{addmargin}[2em]{2em}
Created a isometric strategic game as a part of our Advanced Programming course. Game mechanics are similar to the famous Age of Empires game. Includes LAN multiplayer mode and also a map editor. \href{https://github.com/arminkz/PoorCraft}{(view on Github)}
\end{addmargin}

\pagebreak

\item \textbf{ShaderToy.NET} {C\#, GLSL, OpenGL} 
\begin{addmargin}[2em]{2em}
Implemented a testing and developing environment for GLSL shaders. GLSL is a special code which is executed on GPU to achieve graphical effects. \href{https://github.com/arminkz/ShaderToy.NET}{(view on Github)}
\end{addmargin}


\item \textbf{Ray Casting and Line of Sight Simulator} {Java, Ray Casting}
\begin{addmargin}[2em]{2em}
Implemented two dimensional ray casting algorithm in Java. Used to estimate robot's vision area. \href{https://github.com/arminkz/RayCasting}{(view on github)}
\end{addmargin}


\item \textbf{Sayeh CPU} {VHDL, Hardware Design}
\begin{addmargin}[2em]{2em}
Designed a simple 16-bit SISD CPU using VHDL with a limited instruction set. and also created a basic compiler for the designed architecture. \href{https://github.com/arminkz/ShaderToy.NET}{(view on github)}
\end{addmargin}



\end{rSection} 


\begin{rSection}{Teaching Experience}
\vspace{-1.25em}
\item \textbf{Teaching Assistant} {Data Mining (Under Supervision of Dr. Nazerfard)} \hfill Oct 2019 - Jan 2020
\item \textbf{Teaching Assistant} {Artificial Intelligence (Under Supervision of Dr. Nickabadi)} \hfill Feb 2018 - Jul 2018
\item \textbf{Teaching Assistant} {Advanced Programming (Under Supervision of Dr. Pourvatan)} \hfill Feb 2017 - Jul 2017
\item \textbf{Teaching Assistant} {Fundamentals of Programming (Under Supervision of Dr. Pourvatan)} \hfill Oct 2016 - Jan 2017
\\
\end{rSection} 


%----------------------------------------------------------------------------------------
% make sure to list your skills from most comfortable to least comfortable
% never put anything on here that you cant talk about in an interview
\begin{rSection}{SKILLS}
\begin{tabular}{ @{} >{\bfseries}l @{\hspace{6ex}} l }
Programming Languages & Python, Java, Javascript, C\#, C++, Swift, Kotlin, Ruby, Racket \\
Data Mining \& AI & Tensorflow, Numpy, Pandas, Jupyter Notebook \\
Web Development & Node.js, Angular, SCSS, Flask\\
Mobile Development & Android, iOS \\
Database & MongoDB, MySQL \\
Graphics and Visualization & GLSL, OpenGL, WebGL, Processing\\
Embeded Systems \& Hardware & Arduino, Raspberry Pi, VHDL \\
Other & Git, Docker, LaTeX

\\
\end{tabular}\\
\end{rSection}


%----------------------------------------------------------------------------------------
\begin{rSection}{LANGUAGE SKILLS}

{\bf English}, IELTS
\begin{addmargin}[1em]{0em}
\textbf{Overall: (7.5 / 9): }  Listening: (8.5 / 9) | Reading: (8 / 9) | Writing: (6.5 / 9) | Speaking: (7.5 / 9) 
\end{addmargin}
{\bf Persian}, Native \\
{\bf Azarbaijani}, Native \\


\end{rSection}


\begin{rSection}{HONORS AND AWARDS}

\begin{itemize}
    \itemsep -3pt {}
	\item Appointed as team leader and mentor for Amirkabir University Rescue Robotics Team (Team SOS) \hfill 2018
    
	\item Ranked in top 0.2\% among all students in university entrance exam (Approximately
250000 applicants) in Math. and Eng. \hfill 2015    
    
    \item Awarded Khwarizmi Young Award (KYA) for Developing a novel Mathematical Modeling software. \\ Achieved 6th place among all participants. \hfill 2014
 \end{itemize}

\end{rSection}

\pagebreak

\begin{rSection}{Hobbies}

Hiking, Skiing, Rock climbing, Camping, 3D Printing, Playing board-games (especially Chess) and Watching movies
\\



\end{rSection}


\begin{rSection}{References}

\begin{multicols}{3}
\textbf{Saeed Shiry Ghidary, Ph.D.}\\
Assistant Professor\\
Staffordshire University | Stoke-on-Trent, United Kingdom\\
\href{mailto:saeed.shiryghidary@staffs.ac.uk}{saeed.shiryghidary@staffs.ac.uk}
\\
\\
\textbf{Ehsan Nazerfard, Ph.D.}\\
Assistant Professor\\
Amirkabir University of Technology | Tehran, Iran\\
\href{mailto:nazerfard@aut.ac.ir}{nazerfard@aut.ac.ir}
\\

\textbf{Ahmad Nickabadi, Ph.D.}\\
Assistant Professor\\
Amirkabir University of Technology | Tehran, Iran\\
\href{mailto:nickabadi@aut.ac.ir}{nickabadi@aut.ac.ir}
\\
\end{multicols}

\end{rSection}


\end{document}